% description of block structure 

% block diagram
\begin{figure}[h]
    \centering
    
\begin{tikzpicture}[scale=0.67, every node/.style={transform shape}]
    % macro block diagram for simulation

    \node[block, minimum width=22cm ] (cpp) at ( 2,  0.9) {C++ Wrapper};
    
    % blocks part of the display super-module
    \node[block, minimum width=22cm ] (main) at ( 2, -2.5) {Verilog Wrapper};
    \node[block] (sync) at (-7, -6) {Synchronization module};
    \node[block] (vga) at (-3, -6) {VGA Controller};
    \node[block] (fr) at (-3, -9) {Framer};
    
    % blocks part of the logic super-module
    \node[block, minimum height=5cm] (int) at ( 3, -7.5) {Interface Module};
    \node[block] (tetgen) at ( 7, -6.5) {Tetrimino Generation};
    \node[block] (motcon) at ( 7, -8.5) {Motion Control};
    \node[block, minimum height=5cm] (val) at ( 11, -7.5) {Validation Module};

    \node[draw,inner xsep=4mm,inner ysep=7mm,fit=(vga)(fr)(sync), label={-90:Display Control}] (d) {};
    \node[draw,inner xsep=4mm,inner ysep=7mm,fit=(int)(tetgen)(motcon)(val), label={-90:Game Logic}] (l) {};
    
    \node[draw,inner xsep=4mm,inner ysep=7mm,fit=(main)(d)(l), label={-90:Verilog}] (v) {};

    % super-module connections
    \draw[->] (cpp)-- (main);
    \draw[<->] (vga)-- (vga|- main.south);
    \draw[<->] (int)-- (int|- main.south);

    % display connections
    \draw[<->] (vga.west)-- (sync.east|- vga.west);
    \draw[<->] (fr)-- (vga);

    % logic connections
    \draw[<->] (tetgen)-- (int);
    \draw[<->] (motcon)-- (int);
    \draw[<->] (tetgen)-- (val);
    \draw[<->] (motcon)-- (val);

\end{tikzpicture}

    %\captionsetup{justification=centering}
    \caption{Block diagram $\sankalp{Better\ caption?}$}
\label{fig:blockdiag}
\end{figure}

% verilog
\subsection{Verilog Modules}

% wrapper
\subsubsection{Wrapper}


% logic supermodule
\subsubsection{Logical Modules}

\paragraph{Interface Module}
\paragraph{Tetrimino Generation Module}
\paragraph{Motion Control Module}
\paragraph{Validation Module}

% display supermodule
\subsubsection{Display Modules}

\paragraph{VGA Control Module}
\paragraph{Framer Module}
\paragraph{Synchronisation Module}


% c++
\subsection{C++ Wrapper}

In the absence of an FPGA to run described modules, 
we had to look to other options. With a strong desire to 
maintain real-time playability, we tried several options ---
standard simulations had to be rejected due to not being 
anywhere close to real-time, with the inability to accept inputs.
Having considered the idea of running a simulated clock, we then
moved to forcefully driving a clock for the simulation. We attempted
to use \(\texttt{verilog-vga-simulator}\) \cite{vga-simulator},
a C-based program interfacing with the simulation via 
Verilog Procedural Interface (VPI) \cite{vpi}, drawing the resultant
VGA output via Simple DirectMedia Layer \cite{sdl} referred to as 
SDL herein. We were driven away by a cocktail of issues --- primarily
the performance, combined with the relative inflexibility of VPI 
and the now obsolete \cite{sdl_obsolete} SDL API. However, the cocktail
was more than resolvable.

Saving grace came in the form of Verilator \cite{verilator}, a compiler for
Verilog which translates modules top-down into C++ code with
I/O interfaces handled via classes \cite{verilator-implement}. This allowed
us to bypass having to deal with VPI's idiosyncrasies and limited nature, paired
with the finicky fixes required to obtain a realtime driven clock.

Armed with a way to handle inputs and outputs to our modules in realtime, we 
needed a VGA simulator to handle the outputs. This was executed using
SDL2 \cite{sdl2} and a low-level frame-buffer, rendering to screen via 
a cross-platform hardware accelerated rendering interface (with OpenGL \cite{opengl}
primarily used for testing).

However, a software based simulation is unable to drive a clock fast enough
to run a VGA simulation in any of the common modes \cite{vga_modes}. As such,
a myriad of performance-centric changes had to be made. Alongside several
minor optimizations was a major change to the system --- a drastic reduction in
resolution or frame rate. Realistically, keeping the same resolution, the frame 
rate would have been cut to approximately \(\texttt{1/20 fps}\), far from playable.
The resolution would have to be cut similarly to maintain a playable frame rate.
Since most of our clock cycles are used up driving the VGA controller, keeping true
to the design, we chose to downscale it. Outputting a much smaller resolution from
the simulated Verilog modules and writing an integer scaling \cite{intscale} system
within the relatively fast C++ wrapper. 

We would like to add that much higher raw resolution and frame rates are possible 
and have been seen after removing the limitation to output serial VGA. Providing
the frame buffer as output to the display simulator in parallel reduces the time
complexity of drawing a frame from \(\bigO (n^2)\) to \(\bigO (1)\) in screen 
dimensions. However, having achieved desired levels of responsiveness and 
playability without resorting to bending the limits of the FPGA, we chose not 
to do this.